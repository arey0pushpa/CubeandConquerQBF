% This is samplepaper.tex, a sample chapter demonstrating the
% LLNCS macro package for Springer Computer Science proceedings;
% Version 2.20 of 2017/10/04
%
\documentclass[runningheads]{llncs}
%
\usepackage{graphicx}
% Used for displaying a sample figure. If possible, figure files should
% be included in EPS format.
%
% If you use the hyperref package, please uncomment the following line
% to display URLs in blue roman font according to Springer's eBook style:
% \renewcommand\UrlFont{\color{blue}\rmfamily}

\RequirePackage{amsmath}

\usepackage{verbatim,amssymb,amsfonts,amscd,graphicx}
\usepackage{bussproofs}
\usepackage{color}
\usepackage{times}
\usepackage{placeins}
\usepackage{thmtools}
\usepackage[shortlabels]{enumitem} % added/updated by Ankit to cover no space enum
\usepackage{cite}
\usepackage{url}
\usepackage{lmodern}
\usepackage{booktabs}
\usepackage[justification=justified]{caption}

\spnewtheorem{plaindefinition}{Definition}{\bfseries}{\normalfont}

% GENERAL MACROS
\def\hy{\hbox{-}\nobreak\hskip0pt} \newcommand{\ellipsis}{$\dots$}
\newcommand{\SB}{\{\,} \newcommand{\SM}{\;{:}\;} \newcommand{\SE}{\,\}}
\newcommand{\Card}[1]{|#1|}
\newcommand{\red}{\color{red}}
\newcommand{\blue}{\color{blue}}
\renewcommand{\iff}{\Longleftrightarrow}

% PAPER SPECIFIC MACROS
\newcommand{\var}{\mathit{var}}
\newcommand{\lit}{\mathit{lit}}
\newcommand{\fff}{\varphi}
\newcommand{\matrixf}{\phi}
\newcommand{\dom}{\mathit{dom}}
\newcommand{\qp}{\forall X \exists Y}
\newcommand{\FFF}{F}
\newcommand{\CC}{\mathcal{C}}
\newcommand{\BB}{\mathcal{B}}
\newcommand{\GGG}{\Psi}
\newcommand{\rightOf}{R}
\newcommand{\Neg}[1]{\overline{#1}}
\newcommand{\cc}[1]{Cl(#1)}
\renewcommand{\P}{\mathcal{P}}
\newcommand{\dec}{\varphi_{dec}}
\newcommand{\imp}{\varphi_{imp}}

\newcommand{\AAA}{\mathcal{A}} \newcommand{\BBB}{\mathcal{B}}
\newcommand{\CCC}{\mathcal{C}} \newcommand{\DDD}{\mathcal{D}}
\newcommand{\LLL}{\mathcal{L}} 
\newcommand{\HHH}{\mathcal{H}}
\newcommand{\MMM}{\mathcal{M}} \newcommand{\PPP}{\mathcal{P}}
\newcommand{\QQQ}{\mathcal{Q}}\RequirePackage{amsmath}
\newcommand{\SSS}{\mathcal{S}} \newcommand{\TTT}{\mathcal{T}}
\newcommand{\VVV}{\mathcal{V}} \newcommand{\bigoh}{\mathcal{O}}

%\DeclareMathSymbol{\Omega}{\mathalpha}{letters}{"0A}
%\DeclareMathSymbol{\varOmega}{\mathalpha}{operators}{"0A}
\DeclareMathSymbol{\Omega}{\mathalpha}{operators}{10}
\newcommand\numberthis{\addtocounter{equation}{1}\tag{\theequation}}


\def\preprocess{\mathsf{Preprocess}}
\def\bcp{\mathsf{BCP}}
%\def\split{\mathsf{SplitProblem}}
\def\split{\mathsf{GenerateSubproblems}}
\def\fml{\mathsf{CreateFormula}}
\def\sat{\mathsf{SAT}}
\def\look{\mathsf{LookAhead}}

% Tikz 
\usepackage{tikz}
\usetikzlibrary{tikzmark,decorations.pathreplacing,arrows,shapes,positioning,shadows,trees,shapes.gates.logic.US,arrows.meta,shapes,automata,petri,calc}
\usetikzlibrary{positioning}

% Algo
\usepackage{xspace}
\usepackage{algpascal}
\usepackage[ruled,vlined,linesnumbered,noend]{algorithm2e}
% global change
\SetKwInput{KwData}{Input}
\SetKwInput{KwResult}{Output}
\algnewcommand\TRUE{\textkeyword{true}}
\algnewcommand\FALSE{\textkeyword{false}}
% ----------------- End of the macros

\begin{document}
%
\title{Cube and Conquer for 2QBFs\thanks{Supported by organization.}}
%
%\titlerunning{Abbreviated paper title}
% If the paper title is too long for the running head, you can set
% an abbreviated paper title here
%
\author{Oliver Kullmann\inst{1} \and Ankit Shukla\inst{2} \and Oleg Zaikin\inst{1}}
\authorrunning{O.~Kullmann, A.~Shukla}

\institute{Swansea University, Swansea, UK\\
	\email{O.Kullmann@Swansea.ac.uk, zaikin.icc@gmail.com}
	\and
	Johannes Kepler University, Austria\\
	\email{ankit.shukla@jku.at}
}

\authorrunning{O. Kullmann et al.}
% First names are abbreviated in the running head.
% If there are more than two authors, 'et al.' is used.

\maketitle              % typeset the header of the contribution
%
\begin{abstract}
We present a cube-and-conquer approach to solve 2QBFs.
XXX 
\keywords{Cube and Conqer  \and 2QBF \and Branching Heuristics.}
\end{abstract}
%
%
%
\setcounter{section}{-1}
\section{TODOS}
\label{sec:todos}
\begin{enumerate}
	\item For the completeness in the worst case the algorithm \ref{algo:qbfcc} requires every assignment of universal variables to be used. Can we improve the algorithm?
	\item Use QBF instead of SAT for solving formula from cubes. The cutoff heuristic stops after $k$-number of assignments.
\end{enumerate}

\section{Introduction} \label{sec:introduction}
Cube-and-Conquer \cite{HeuleKWB11} is an effective technique to solve hard SAT instances. The approach consists of two-sequential phases: first, use lookahead techniques to partitions a problem into many thousands (or millions) of cubes. This phase is called \textit{cube-phase}.
Then use a conflict-driven solver to solve the generated subproblems; a \textit{conquer-phase}.

We present a cube-and-conquer approach to solve 2QBFs based on the following \textbf{heuristic}: branch on an existential variable that maximizes the number of pure universal literals.

XXX

\section{Preliminaries} \label{sec:preliminaries}
XXX

\section{Cube and Conquer for 2QBF} \label{sec:cc-qbf}

The algorithm \ref{algo:qbfcc} takes a 2QBF formula as an input and returns whether the formula is \TRUE\space or \FALSE. First, we split the formula $\FFF$ into subformulas $F_1, \dots, F_n$ using the procedure $\split(\FFF)$. Then, each subformula is independently solved using a SAT solver. If any of the subformula is unsatisfiable the algorithm returns \FALSE\space otherwise returns \TRUE.

\begin{algorithm}[]
	\DontPrintSemicolon
	\SetAlgoLined
	\KwData{A 2QBF formula $\FFF = \qp.\matrixf$}
	\KwResult{\TRUE \space or \FALSE}
	%$\FFF^{\prime} = \preprocess (\FFF)$ \;
	$F_1,\dots,F_n = \split (\FFF)$ \;
	$R$ = $\sat(F_1); \dots; \sat (F_n)$  // Run SAT solver in parallel \;
	\eIf{ $\exists r \in R$:
		$\sat(r)$ == \FALSE \;
		}
	{ % 
		return \FALSE 
	} % Close If
	{ % Open Else
	return \TRUE } 
	\caption{Cube and Conquer for 2QBF}
	\label{algo:qbfcc}
\end{algorithm}


The $\split(\FFF)$ procedure partition the input 2QBF formula $\FFF$ into subformulas $F_1, \dots, F_n$. 
The recursive procedure has four inputs, 2QBF formula $\FFF$, DNF formula $\BB$ (cubes; represented as set of literals), the set of decision literals (denoted by $\dec$) and the set of implied literals (denoted by $\imp$). Implied literals are assignments that were forced by BCP or preprocessing techniques. Initially, $\split$ is called with the input formula $\FFF$ and all the other parameters as empty sets. XXX
%The cubes $\BB$ cover all subproblems of $\FFF$ that have not been satisfied during the partition procedure.
%The clauses $C$ are the learnt clauses implied by $F$ and represent refuted branches in the DPLL tree.

\begin{algorithm}[]
	\DontPrintSemicolon
	\SetAlgoLined
	 \KwData{$\FFF = \qp.\matrixf$, DNF $\BB = \dec = \imp = \{\}$ \;}
	 \KwResult{Subformulas $F_1,\dots,F_n$ of the formula $\FFF$}
	$\FFF, \imp$ := $\bcp (\FFF,\dec,\imp)$  \;
	\uIf{ $\dec \cup \imp$\xspace falsify a clause in \FFF}{
	return $<\BB, \CC \cup \{\neg \dec\}>$ } 
	 \uElseIf {cutoff heuristic is triggered} {
	 	return  $\fml(\FFF, \BB \cup \dec, \CC)$} 
	$l$ :=  $\look(\FFF, \dec \cup \imp)$  \;
	$<\BB, \CC>$ := $\split(\FFF,\BB, \CC, \dec \cup \{l\}, \imp)$  \;
	return $\split(\FFF,\BB, \CC, \dec \cup \{ \neg l\}, \imp)$ \;	
	
	\caption{$\split(\FFF,\BB, \CC, \dec, \imp)$ }
	
\end{algorithm}

\section{Minimising the variance}\label{sec:min-var}

Consider a finite probability space $(\Omega, X)$, i.e., a finite set $\Omega = \{\omega_1,\dots, \omega_m\}$ of ``outcomes" together with a probability assignment $X : \omega \to [0, 1]$ such that $\sum_{\omega \in \Omega} X (\omega) = 1$; we assume furthermore that no element has zero probability ($\forall \omega \in \Omega: X(\omega) > 0$).

%The random variable $X^{-1}$ on the probability space $(\Omega, X)$ assigns to every outcome $\omega$ the value
%$X(\omega)^{-1}$. The expected value of $X$ is the number
%of outcomes is:
%
%\begin{align*}
%	E(X^{-1}) =	X(\omega) \cdot X(\omega)^{-1} = |\Omega| \numberthis \label{eqn}
%\end{align*}

\subsection{The variance of the estimation of the number of leaves} \label{subsec:variance}

The variance of the probability assignment $X$ is:
\begin{align*}
\text{Var}(X) &= E((X - E(X))^2) \\  &= E(X^{2}) - E(X)^2 \in \mathbb{R} \geq 0
\end{align*}

To estimate \text{Var}(X) one needs to estimate $E(X^{2})$, that is, the second moment
of $E(X)$. By definition we have the expected value $E(X)$, of the probability assignment $X$: 
\begin{align*}
E(X) &= \sum_{\omega \in \Omega} Pr(\omega) \cdot X(\omega), \textrm{where} \,\, X(\omega) = Pr(\omega^{-1}) \\
&= \sum_{\omega \in \Omega} Pr(\omega) \cdot  Pr(\omega)^{-1}
\end{align*}

Similarly, $E(X^{2})$ is:
\begin{align*}
E(X^{2}) &= \sum_{\omega \in \Omega} Pr(\omega) \cdot  Pr(\omega)^{-2} \\
    &= \sum_{\omega \in \Omega}  Pr(\omega)^{-1}
\end{align*}

The $E(X)$ for our analysis of branching trees is number of leaves in the tree. i.e., $E(X) = \#lvs(T)$

%The variance of the random variable $X^{-1}$:
%\begin{align*}
%\text{Var}(X^{-1}) &= E((X^{-1} - \#lvs(T))^2) \\  &= E((X^{-2})) - \#lvs(T)^2 \in \mathbb{R} \geq 0
%\end{align*}


\subsection{Optimising the distance depend on a parameter} \label{optimsing-d}

Given a rooted tree $(T,r)$. A special (interesting) case is where distance $d$ is given as a convex linear
combination of distances d1, . . . , dk (that is, d =
Pk
i=1 λi
· di for λi ≥ 0 and
Pk
i=1 λi = 1). 

\section{Related Work} \label{sec:related-work}
A parallel algorithm for solving 2-QBF was presented in \cite{AspvallLLS96} based on the implication graph.

The concept of dividing the search space into 2 or more disjoint parts is used in parallel QBF solving. A parallel QBF solver consists of multiple copies of a sequential solver and each individual solver is given a small part of the original problem.
A naive way to divide the input formula is to solve each path in the search tree using a different solver. Example of such solvers XXX

A variant of the guiding path method used in SAT solving
has been found effective at generating subproblems in parallel QBF solving. In this method a sequential solver instance in a separate computing node is provided with a set of assumptions. Assumptions are predefined variable assignments that the solver has to take into account in the solving process. A Master/Slave Message Passing Interface (MPI) based design is used where master process generates sets of assumptions and distributes them to the solver instances running on the computing nodes.

The first work was of PQSolver \cite{FeldmannMS00}, a distributed theorem-prover for QBFs. Initially, processor 0 starts its work on the input formula. All other processors are idle. Idle processors send requests for work to a randomly selected processor. 

PQSolver was based on the basic DPLL algorithm, without conflict analysis, solution analysis, watched literals, and many
other advanced techniques. PaQuBE 
\cite{LewisMSNBG09} rectify these shortcomings. XXX
% ---- Bibliography ----
%
% BibTeX users should specify bibliography style 'splncs04'.
% References will then be sorted and formatted in the correct style.

 \bibliographystyle{splncs04}
 \bibliography{Bibliographie}


\end{document}
